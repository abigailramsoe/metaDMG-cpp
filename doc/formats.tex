\documentclass[10pt]{article}
\usepackage{color}
\definecolor{gray}{rgb}{0.7,0.7,0.7}
\usepackage{framed}
\usepackage{enumitem}
\usepackage{longtable}
\usepackage[pdfborder={0 0 0},hyperfootnotes=false]{hyperref}

\addtolength{\textwidth}{3.4cm}
\addtolength{\hoffset}{-1.7cm}
\addtolength{\textheight}{4cm}
\addtolength{\voffset}{-2cm}


\begin{document}

\title{metadamage(?) formats}
\author{tsk}
\maketitle
\vspace*{1em}

This document describe some of the internal formats used by the metadamage software. These are at the current time.
\begin{itemize}
\item .bdamage.gz
\item .lca
\item .stat
\item .dfit.txt.gz
\item .dfit.stat.txt.gz
\end{itemize}

\section{bdamage format}
bdamage files are files that contain counts of mismatchs conditional on strand and cycle (position within read). These are generated with metadamage lca or metadamage getdamage. 
The first 8 bytes magic number determines which bdamage version. If no magic number is present then version0 is assumed. 

\subsection{version 0}
First version of the bdamage file is a single bgzf compressed file. MAXLENGTH occurs once in the beginning of the file, followed by succesive blocks of data[1-8]. Block[3-5] indicates the actual mismatch counts for the forward which we have MAXLENGTH times. Block[6-8] indicates the actual mismatch counts for the reverse strand which will also occur MAXLENGTH times. 
\begin{table}[h!]
\begin{tabular}{rllll}
  \hline
  {\bf Col} & {\bf Field} & {\bf Type} & {\bf Brief description} \\
  \hline
  0 & {\sf MAXLENGTH} & int &  Number of cycles\\\hline
  1 & {\sf ID} & int &  Id for mismatch type$^1$\\
  2 & {\sf NREADS} & size\_t & Number of reads used supporting the mismatch matrix\\\hline\hline
  3 & {\sf 1} & int[16] & mismatch rate for first cycle from the 5prime\\
  4 & {\sf $i$} & int[16] & mismatch rate for the $i$'th cycle from the 5prime\\
  5 & {\sf MAXLENGTH} & int[16] & mismatch rate for the last cycle from the 5prime\\\hline
  6 & {\sf 1} & int[16] & mismatch rate for first cycle from the 3prime\\
  7 & {\sf $i$} & int[16] & mismatch rate for the $i$'th cycle from the 3prime\\
  8 & {\sf MAXLENGTH} & int[16] & mismatch rate for the last cycle from the 3prime\\\hline
  \hline
\end{tabular}\label{tab1}
\caption{Content of bdamage.gz file. Note$^1$ This is either the taxidID or the referenceID relative to the SAM/BAM header for single species resequencing projects or it is the \textit{taxid} if output has been generated with metadamage lca 3) Order is given by AA,AC,AG,AT,CA,CC,CG,CT,GA,GC,GG,GT,TA,TC,TG,TT, with first base indicatting reference nucleotide and second base indicating observed nucleotide}.
\end{table}
 
\clearpage
\section{.lca}
This section describes the test output generated by a metadamage lca subfunctionality and contains information at the readlevel regarding both taxonomic information and statistics pertaining usefull readinformation.\\

First line of the file begins with a hashtag followed by the actual command used for generating the file. Last entry of the line is again a hashtag followed by the git commit id which will serve as a primitive versioncontrol.

Each line consists of a number of items seperated by tabspace. First entry contains readID together with other information seperated by colon. After the first entry succesive blocks of the type taxid:name:''taxlevel'' from the lca toward the root. The complete specification is seen in table below.

\begin{table}[h]
\begin{tabular}{rll}
  \hline
  {\bf Col} & {\bf Brief description} \\
  \hline
  1 & {\sf readID} & readID, this might contains colon\\
  2 & {\sf seq} & The actual sequence\\
  3 & {\sf length(seq)} & The length of the sequence\\
  4 & {\sf nAlignments} & The number of alignments used for inferring the lca\\
  5 & {\sf gc-content} & The GC content for the sequence\\\hline\hline
  6 & {\sf taxid} & the taxomic id (integer)\\
  7 & {\sf taxid} & the taxonomic name(string)\\
  8 & {\sf ``taxlevel''} & the taxonomic level\\\hline\hline      
\end{tabular}\label{tab2}
\caption{Content of a .lca file. Note that 1) seperate between field[1-5] is colon, but the readID might also contain colon. 2) the quotes around field[8] is intentional since taxlevels might contain spaces. 3) Not that the number of tab seperated entries is different between reads since this is the number of nodes needed to traverse from lca to root}
\end{table}
\section{.stat}\label{sec:stat}
Very simple tabseperated flatfile
\begin{enumerate}
\item taxid
\item Number of supporting reads
\item Mean lengths of supporting reads
\item Variance of the lengths of the supporting reads
\item Mean gccontent of supporting reads
\item Variance of the gccontent of supporting reads
\item name of lca in quotes (if relevant, otherwise NA)
\item name of taxomic level of lca in quotes (if relevant, otherwise NA)
\end{enumerate}
\clearpage
\section{.dfit.txt.gz}
This output format was added 8d28d337 Sun Jul 23.  Depending on
runmode and parameters supplied to the program. It will contain the
per file, the per reference or the per species estimate of damage.

\begin{table}[h]
\begin{tabular}{rll}
  \hline
  {\bf Col} & {\bf Brief description} \\
  \hline
  1 & {\sf id} & identifier see paragraph for details\\
  2 & {\sf A} & Dfit statistic. Damage at position one, taking into account offset\\
  3 & {\sf q} & per cycle decrease\\
  4 & {\sf c} & background substitution rate or noise baseline\\ 
  4 & {\sf $\phi$} & \\
  5 & {\sf llh} & Likelihood for our MLE\\
  6 & {\sf nopt} & Number of function calls used for obtaining our MLE\\
  7 & {\sf Zfit} & Z value\\
  8 & {\sf Zconf} & significance\\\hline\hline
  9 & {\sf $K_i$} & Number of CT og AG observations\\
  10 & {\sf $N_i$} & Count of either CA,CC,CG,CT or AA,AG,AC,AT observations\\
  11 & {\sf $Dx_i$} & Fitted value of cycle specific damage\\
  13 &{\sf $Dconf_i$ }& Confidence interval for the cycle specfic damage\\\hline\hline     
\end{tabular}\label{tab3}
\caption{Content of a .dfit.txt.gz file. Note that entry nine to 13 is
  repeated for each cycle first of the 5\' and then from the 3\'. With
  the total number of times repeated is given by \ref{tab1}.} 
\end{table}
Depending on which parameters that was supplied to both
\emph{getdamage} and \emph{dfit}, the content of the id will be
different. If a bamfile is supplied (with -bam) then each line will be
the information associated with the different refids in the bam file
and the id will be the referenceids from the bam file. The case
scenario for this would be either obtaining perchromosome estimates of
damage or per reference damage which could be relevant for metagenomic
studies. If user are computing the damagesignal in the context of the
lca. Then the id column will contain the taxid. If -names has been
supplied to the dfit program, then the id column will the
taxid:\emph{scientific name} . If -nodes has not been defined the
dfit.txt.gz will only contain information for the observed
references. If -nodes has been defined the program will aggregate the
summary statitics for the internal nodes.

\section{.dfit.stat.txt.gz}
This output format was added 8d28d337 Sun Jul 23.  Depending on
runmode and parameters supplied to the program. It will contain the
per file, the per reference or the per species estimate of the
statistic. This format extends \ref{sec:stat} to include the internal
nodes.
\end{document}

%%% Local Variables:
%%% mode: latex
%%% TeX-master: t
%%% End:
